\documentclass[12pt,a4paper]{article}% 文档格式
\usepackage{graphicx} % Required for inserting images
\usepackage{ctex,hyperref}% 输出汉字
\usepackage{amsmath} 
\usepackage{amssymb}
\usepackage{framed}  % 用于创建边框
\usepackage{enumitem}
\usepackage{algorithm,algorithmicx,algpseudocode,amsmath,amssymb}
\usepackage{amsfonts,geometry}
\usepackage{booktabs}
\floatname{algorithm}{算法}
\renewcommand{\algorithmicrequire}{\textbf{输入:}}
\renewcommand{\algorithmicensure }{\textbf{输出:}}
\newenvironment{solution}
  {\begin{framed}\textbf{Solution:}\par}
  {\end{framed}}

\title{人工智能与机器学习基础 2025-HW3}
\author{TA: 王珏}
\date{November 2025\\Deadline : 2025/12/31 23:59}

\begin{document}

\maketitle

\section{支持向量机(Support Vector Machine)}

支持向量机(SVM)通过最大化分类间隔(Margin)来获得泛化能力强的决策边界。在高维空间中,通过核函数(Kernel Function)可实现非线性分类。

\paragraph*{(a)(15分) 硬间隔 SVM}

给定线性可分数据集,SVM 的目标为:
\[
\min_{w,b} \frac{1}{2}\|w\|^2 \quad s.t.\quad y_i(w^\top x_i + b) \ge 1
\]

请完整复现上课推导步骤,说明如何得到以上目标。


\paragraph*{(b)(10分) 软间隔 SVM} 硬间隔的定义要求要把正负类点完全分开,这个
要求在数据集含有噪音点时可能过于严格,为了解决该问题,我们引入松弛因子$\zeta_i$,有
\[
y_i(w\cdot x_i + b) \ge 1 - \zeta_i
\]

相应的,新的目标函数变为:
\[
\min_{w,\, b,\, \zeta} \quad \frac{1}{2}\|w\|^{2} \;+\; C \sum_{i=1}^{N} \zeta_i^{2}
\]

\[
\text{s.t.}\quad y_i (w \cdot x_i + b) \ge 1 - \zeta_i,\quad \forall i = 1,2,\ldots,N,
\]

\[
\zeta_i \ge 0,\quad \forall i = 1,2,\ldots,N.
\]

\medskip

试写出其对偶形式。


\paragraph*{(c)(15分) 支持向量机求解}

考虑如下 3 个二维样本,其中两正一负:
\[
X_1 = (3,3),\quad y_1 = +1;\qquad
X_2 = (4,3),\quad y_2 = +1;\qquad
X_3 = (1,1),\quad y_3 = -1.
\]

我们希望通过硬间隔 SVM 求得最大间隔分离超平面。其对偶优化问题为:
\[
\max_{\alpha}\; 
\sum_{i=1}^{3} \alpha_i 
- \frac12 \sum_{i=1}^{3}\sum_{j=1}^{3} 
\alpha_i\alpha_j y_i y_j (x_i^\top x_j),
\]
满足约束:
\[
\alpha_i \ge 0,\qquad i=1,2,3,
\]
\[
\alpha_1 y_1 + \alpha_2 y_2 + \alpha_3 y_3 = 0.
\]

\begin{enumerate}[label=(\arabic*)]

\item 写出该对偶问题的显式形式(不消元)。

\item 利用约束消去变量,写出化简后的二元对偶目标函数。

\item 对该对偶问题进行求解,给出最优对偶变量
$(\alpha_1^{\*},\alpha_2^{\*},\alpha_3^{\*})$.

\item 根据最优对偶变量,利用
\[
w = \sum_i \alpha_i y_i x_i
\]
求出原始空间中的法向量 \(w\)。

\item 选取任意一个支持向量,求出偏置 \(b\)。

\item 写出最终分类超平面方程,形如
\[
w^\top x + b = 0.
\]
\end{enumerate}

\paragraph*{(c)(5分) 思考}

\begin{enumerate}[label=(\arabic*)]
\item 由此题思考\textbf{解释支持向量在模型中的作用}:为什么只有支持向量会影响最优超平面?(尽量用自己的语言表述,提示:拉格朗日对偶条件的互补松弛性)
\end{enumerate}


\section{决策树(Decision Tree)}

决策树是一种可解释性强的监督学习算法,它通过递归划分特征空间,将复杂的决策问题分解为一系列“如果–那么”的规则。核心问题是如何选择最优划分特征,常见准则包括信息增益(Information Gain)和基尼指数(Gini Index)。
\paragraph*{(a)(10分) 决策树分裂特征选择与分类错误率计算}

我们希望利用决策树来预测贷款申请人是否违约(1 表示违约,0 表示未违约)。给定如下训练样本(与课堂案例一致):

\begin{center}
\begin{tabular}{cccc}
\toprule
编号 & 收入水平 & 是否有房 & 是否违约 \\
\midrule
1 & 高 & 是 & 0 \\
2 & 高 & 否 & 0 \\
3 & 中 & 是 & 0 \\
4 & 低 & 否 & 1 \\
5 & 低 & 否 & 1 \\
\bottomrule
\end{tabular}
\end{center}

\begin{enumerate}[label=(\arabic*)]

\item 课堂上我们使用“分类错误率(classification error)”来评估一次分裂是否有效。请分别以收入水平
和是否有房为分裂特征,在根节点上构建一个决策树桩(decision stump),并计算每种分裂方式的分类错误率。请写出你的计算过程,并根据错误率选择根节点的最优分裂特征。

\item 课堂中我们提到:由于分类错误率可能对部分“看似有效但局部不稳定”的分裂不敏感,因此需要额外的停止条件或剪枝(pruning)策略以避免过拟合。请结合课堂内容简述:
\begin{itemize}
    \item 为什么仅用分类错误率作为分裂标准可能导致过拟合?
    \item 提前停止(early stopping)与剪枝(pruning)分别如何缓解这一问题?
\end{itemize}

\item 假设将“收入水平”替换为一个连续特征“年收入(万元)”。根据课堂讲述的“阈值分裂(threshold split)”,回答:
\begin{itemize}
    \item 如何确定候选阈值?
    \item 为什么只需要考虑相邻样本之间的中点?
    \item 这个年收入特征能否反复使用?
\end{itemize}

\item 为了更好地区分不同分裂的“纯度改善”,ID3 使用信息熵(entropy)与信息增益(information gain):
\[
\mathrm{Gain}(D, A) = H(D) - H(D|A)
\]
请计算上述两个特征(收入水平、是否有房)的信息增益,并比较它们与分类错误率的选择结果是否一致。

\item 信息增益容易偏好“取值数较多”的特征(over-favor multi-valued attributes)。  
C4.5 通过引入“增益率(gain ratio)”解决此问题:
\[
\mathrm{GainRatio}(A) = \frac{\mathrm{Gain}(D, A)}{\mathrm{IV}(A)}
\]
其中 $\mathrm{IV}(A)$ 是“固有值”(intrinsic value)。请解释:
\begin{itemize}
    \item 为什么信息增益会偏好取值较多的特征?
    \item 增益率是如何缓解这一问题的?
\end{itemize}
由于课堂上未提及,此处补充两个算法的介绍供大家参考:\href{https://zhuanlan.zhihu.com/p/139188759}{C4.5算法介绍},\href{https://zhuanlan.zhihu.com/p/133846252}{ID3算法介绍}

\end{enumerate}


\paragraph*{(b)(10分) 剪枝策略与过拟合}

请解释:
\begin{enumerate}[label=(\arabic*)]
\item 为什么决策树容易过拟合?
\item 简述“预剪枝(Pre-pruning)”与“后剪枝(Post-pruning)”的主要区别与适用场景。
\item 若你要在小样本任务中使用决策树模型,剪枝策略应如何调整?请说明理由。
\end{enumerate}

\section{集成学习(Ensemble Learning)}

集成学习通过组合多个弱学习器(如决策树)以提升性能。常见方法包括 Bagging、随机森林(Random Forest)与 Boosting 系列(如 AdaBoost、XGBoost)。

\paragraph*{(a)(15分) Bagging 与 Boosting 的比较与推导}

\begin{enumerate}[label=(\arabic*)]
\item 试解释 Bagging 与 Boosting 在“样本采样方式”和“模型训练目标”上的根本区别。
\item 假设我们使用 AdaBoost 算法,在第 $t$ 轮弱分类器的加权错误率为 $\epsilon_t$,请证明其权重系数:
\[
\alpha_t = \frac{1}{2}\ln\frac{1-\epsilon_t}{\epsilon_t}
\]
并计算该权重系数的范围,解释该公式反映的直觉含义。
\end{enumerate}

\paragraph*{(b)(10分) 随机森林与偏差-方差权衡}

假设你训练了一个包含 100 棵树的随机森林,每棵树的误差率为 0.2,且不同树之间错误独立。
\begin{enumerate}[label=(\arabic*)]
\item 请计算整体模型错误率的期望值(可用二项分布近似)。
\item 若树与树之间相关性增大,会对整体性能产生何种影响?请结合“偏差–方差分解”进行分析。
\end{enumerate}

\paragraph*{(c)(10分) 问答题}

判断以下说法是否正确并给出理由:
\begin{enumerate}[label=(\arabic*)]
\item 有效的集成学习需要集合中的模型具有单一性,最好将同一类型的预测模型结合起来。
\item 训练集成模型时,单个模型的参数不会随之更新。
\item AdaBoost算法中,需要按照之前学习器的结果对训练数据进行加权采样。
\item 尝试解释下课堂上所讲的:\textbf{Random forest does not perform as well in general as boosting.}(随机森林在一般情况下性能不如boosting)。
\end{enumerate}


\end{document}
